\documentclass[10pt,a4paper,french]{article}
\usepackage{babel}
\usepackage[utf8]{inputenc}
\usepackage[T1]{fontenc}
\usepackage{../clovisai}

\begin{document}

\title{Cours de droit}
\author{Clovis AI}
\maketitle

\begin{abstract}
Ceci est l'introduction du document.
\end{abstract}
\tableofcontents

\part{Introduction}

\section{La règle de droit}

\subsection{La finalité de la règle de droit}

\subsubsection{Idée de sous-sous-partie}

\hrefu{https://google.com}{Go to google!}

\cref{test}

\cref[Un autre titre]{sources}

\cf{sources}

\cnref{sources}

\cnpref{sources}

Plus d'information à la \autoref{test}.

Vous pouvez aussi voir \autoref{test2}.

La règle de droit, aussi appelée loi, a comme but d'organiser la vie en société et les relations \cf{sources} entre les membres qui l'a composent. Les buts principaux sont d'assurer la sécurité des personnes (le code pénal), la protection des salariés (le code du travail) et la sécurité des biens (pénalisation du vol), mais aussi l'organisation économique (droit des marchés, droit du commerce), le gouvernement, l'organisation sociale. $f(x) = x^2$; $f: \mathbb{R} \rightarrow \mathbb{R}$. $bonjour$, $\mathit{bonjour}$. On peut utiliser $\setN$. On peut aussi utiliser $\parmi{p}{n}$ et $\arrang{p}{n}$. Mais aussi $\setI \union \setR = \setC$. $a \land \lnot b \infinity$ et $\setps{N}$. $\setN$.

\subsubsection{Comme vous pouvez le voir, je n'ai plus d'idées}

\subsection{Caractère général}

La règle de droit est une règle de conduite, qui s'applique à tout le territoire pour tout ce qui s'y produit. Elle est impersonnelle (elle n'admet pas de cas particuliers).

La règle de droit est un commandement qui doit être respecté. Il en existe deux sortes; les règles impératives (qui ordonnent ou interdisent une conduite), et les règles supplétives.

\subsection{Droit objectif et droit subjectif}

Le droit objectif est un ensemble de règles de droit régissant la vie en société, les rapports entre particuliers, le fonctionnement de l'État, son rapport avec les citoyens. Le droit objectif pose parfois des interdictions.

Le droit subjectif est une prérogative possédée à l'égard de biens ou de personnes. Il est reconnu par le droit objectif qui tachera d'en déterminer la nature et l'étendue.

\section{Les branches du droit}

Les branches du droit sont un classement des règles de droit en fonction de l'objet ou du domaine.

\subsection{Droit privé et droit public}

Le droit public définit les rapports entre l'État et les particuliers. Il est séparé entre le droit constitutionnel et le droit administratif.

Le droit privé définit les rapports entre particuliers, il est séparé entre le droit civil (rapports entre personnes privés en l'absence de règles spécifiques) et le droit commercial (le statut des commerçants et des sociétés).

On retrouve aussi le droit mixte, c'est-à-dire le droit pénal, qui a pour rôle de proposer la peine.

\subsection{Droit national et international}

Le droit national réglemente les rapports sociaux à l'intérieur d'un État et le droit international réglemente les relations entre plusieurs États, gérés par des traités (accords conclus entre différents États)\label{test}.

Le droit international est divisé en trois branches:
\begin{itemize}
\item Le droit communautaire (eg. l'UE),
\item Le droit international public (les rapports entre les États et les compétences des organisations internationales)\label{test2},
\item Le droit international privé (rapports entre ressortissants qui relèvent d'États différents).
\end{itemize}

\section{Les sources du droit}\label{sources}

Dans le droit ancien, les coutumes servaient de règles, avec les ordonnances du Roi et le droit canonique (l'Église).

\subsection{La constitution}

La constitution peut être définie comme un ensemble de règles fondant l'autorité étatique, organisant ses institutions, lui donnant ses pouvoirs et imposant des limitations. Elle vient garantir les libertés des citoyens. Notre constitution a été écrite le 4 octobre 1958 et a été modifiée plusieurs fois (par exemple lors du passage du quinquennat au septennat).

\subsection{Les autres sources du droit}

\subsubsection{La loi}

Les lois interviennent dans tous les domaines du pouvoir législatif, énumérées par l'article 34 de la constitution. Une loi est votée par le parlement (l'Assemblée Nationale et le Sénat). Le pouvoir exécutif peut avoir un pouvoir normatif: règlements, décrets, arrêtés. C'est la constitution qui définit les domaines de compétences.

\subsubsection{La jurisprudence}

C'est l'ensemble des décisions rendues par des juridictions (les tribunaux), soit dans une matière, soit dans une branche du droit. Elle désigne une solution donnée par les juridictions à un problème donné, dont on pourra s'inspirer pour les problèmes futurs similaires.

\subsubsection{Hiérarchie des normes}

Pour empêcher une contradiction entre différentes sources de loi, le 30 octobre 1998, il a été déclaré que la constitution est supérieure aux traités -- mais l'Union Européenne considère que le droit communautaire est supérieur.

\part{Personnes et patrimoine}

\section{La personne juridique}

Les droits subjectifs (vie, vote, propriété) sont attribués aux personnes juridiques.

\subsection{La personne physique}

Tout être humain est une personne physique, et est donc une personne juridique, de sa naissance à sa mort.

La mort est certaine lorsque l'arrêt complet et irréversible a été constatée. Selon 1232-1 du code de la Santé Publique, <<La mort est une absence totale de conscience et d'activité motrice spontanée, abolition de tous les réflexes du tronc cérébral et absence de respiration spontanée.>> Tant qu'un décès n'est pas établi, une personne disparue est considérée vivante, donc, digne de droits.

Pour une absence, on présume que la personne est vivante: toute personne peut saisir le juge pour faire déclarer la présomption d'absence, qui s'arrête au retour ou à la preuve de la mort.

À défaut, on peut déclarer une absence (au tribunal de Grande Instance du dernier lieu de résidence) qui ne peut être déposée que 9 ans après la présomption d'absence, ou 20 ans si elle n'a pas été déclarée. Ce jugement a pour valeur d'acte de décès.

Une personne `disparue' correspond à une quasi-certitude de mort quand le décès ne peut pas être constaté par manque de cadavre.

\subsection{La personne morale}

Il s'agit de regroupements de personnes autour d'un intérêt commun. On en trouve deux catégories: publiques (l'État, les départements, les écoles, les hôpitaux...) et privées (entreprises et associations).

Une personne morale est créée lors d'un enregistrement auprès du RCS (on peut vérifier que le nom n'est pas déjà pris auprès de l'INPI), elle disparaît lors de la cessation d'activité ou la faillite.

\section{Identification des personnes}

À la naissance de l'enfant, un acte d'état civil est établi dans lequel l'enfant est désigné comme masculin ou féminin. Les transsexuels sont des gens nés d'un sexe mais développant un caractère du sexe opposé -- la Cours de Cassation a refusé les changements de sexe comme étant des cas de <<déviance sexuelle>>, mais a été condamnée par commission européenne des droits de l'homme.

L'identité d'une personne est constituée de son prénom et de son nom de famille, le nom est immuable (sauf intérêt légitime comme les consonances ou l'historique). Pour les personnes morales, on l'appelle la <<dénomination sociale>>, qui est choisie librement parmi celles qui ne sont pas utilisées.

Le domicile est un autre moyen d'identification, on ne peut en avoir qu'un (qu'on distingue du lieu de résidence ou de logement), c'est une obligation du code civil (même pour les SDF). Il détermine le tribunal, les listes électorales, le lieu de mariage... (pour les personnes morales, l'équivalent est le siège social).

\part{Organisation juridictionnelle}

\part{Mécanisme de responsabilités}

\part{Principes directeurs en droit du travail}

\end{document}
