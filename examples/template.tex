\documentclass[10pt,a4paper,french]{book}
\usepackage{babel}
\usepackage[utf8]{inputenc}
\usepackage[T1]{fontenc}
\usepackage{../clovisai}

\makeindex
\makeglossaries

\glossaryentry{lorem}{Lorem Ipsum}{Lorem Ipsums}{Ceci est la description.}
\newacronym{fps}{FPS}{Frame Per Second}
\newdualentry{cgi}{CGI}{Computer-Generated Interface}{This is the description}

\begin{document}

\title{Titre du document}
\author{Auteur du document}
\maketitle

%\begin{abstract} % ARTICLE ONLY
%Ceci est l'introduction du document.
%\end{abstract}

\tableofcontents

\frontmatter % Introduction, BOOK ONLY
\chapter{Prologue}

\mainmatter % Corps du texte, BOOK ONLY

\part{Première partie du livre}
\entry{test}
\chapter{Premier chapitre}
\section{Texte}

\part{Deuxième partie du livre}
\chapter{Une autre chapitre}
\section{Autre}
\gls{lorem} \gls{fps} \gls{fps} \gls{cgi}

\appendix % Annexes, ARTICLE & BOOK
\part{Annexes}

\chapter{Remerciements}

\backmatter % Index, bibliographie, BOOK ONLY

\bibliography{•} % THE .BIB FILE HERE, WITHOUT THE EXTENSION
\cprintindex
\cprintglossaries

\end{document}
